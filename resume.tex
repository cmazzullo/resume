% resume.tex
%
% This is the definitive source code of my resume, all of the other
% files in this folder are compiled from it.

\documentclass[11pt,a4paper,sans]{moderncv}
\moderncvstyle{banking} % style options: 'casual', 'classic', 'oldstyle', 'banking'
\moderncvcolor{blue}
\usepackage[scale=0.75]{geometry} % adjust the page margins


% PERSONAL DATA

\name{Christopher}{Mazzullo}
% \title{Resumé title}                               % optional, remove / comment the line if not wanted
% \address{9584 Burnt Oak Drive}{Fairfax Station, VA 22039}{}% optional, remove / comment the line if not wanted; the "postcode city" and "country" arguments can be omitted or provided empty
\phone{(571)~263~4291}                   % optional, remove / comment the line if not wanted; the optional "type" of the phone can be "mobile" (default), "fixed" or "fax"
\email{chris.mazzullo@gmail.com}
% \homepage{cmazzullo.github.com}                         % optional, remove / comment the line if not wanted
\social[github]{cmazzullo.github.com}


% CONTENT

\begin{document}

\makecvtitle

\section{Objective}

 \cvitem{} { Developer with 2 years of experience building and
   deploying software for federal agencies looking to leverage
   programming skills in an entry level position at Information Management Services, Inc.  }

\section{Education}
\cventry{May 2015}{\emph{BS} Physics, Minor Computer Science}{George Mason University}{Fairfax, VA}{}{}

\section{Experience}
\cventry{Sept 2015--Nov 2016}{EOP Engineer}{US Naval Observatory}{Washington, DC}{}{
\begin{itemize}
\item Improved Fortran 95 codebases responsible for computing
  Earth orientation parameters, the results of which are used in Navy GPS systems.
\item Overhauled and updated the public facing websites of the Earth
  Orientation division at USNO.
\item Automated processes using Python and Bash to generate data products regularly and
  without intervention.
\end{itemize}}

\cventry{June 2014--Sep 2015}{Python Developer}{US Geological Survey}{Reston, VA}{}{
\begin{itemize}
\item Developed Python programs to parse large amounts of raw data from pressure sensors in the Atlantic.
\item Applied Fourier analysis and linear wave theory to implement algorithms that convert water pressure readings into statistical data about wave height and frequency.
\end{itemize}}

\cventry{May 2014--May 2015}{Research Assistant}{George Mason University}{Fairfax, VA}{}{
\begin{itemize}
\item Developed software in IDL to analyze the Hinode satellite's images with funding from the Undergraduate Research Scholars Program, an internal grant at Mason.
\item Investigated the velocity of plasma in coronal loops to help discriminate between diferent models of coronal heating, and presented the research at the 2015 National Conference on Undergraduate Research.
\end{itemize}}


\cventry{Aug 2013--May 2014}{Teaching Assistant}{George Mason University}{Fairfax, VA}{}{
\begin{itemize}
\item Graded papers and assignments for a first year Computer Ethics class.
\item Held regular office hours to tutor students struggling with the course.
\end{itemize}}

\section{Skills}
\cvlistitem{Proficient with Python, Java, C, Bash and Fortran.}
\cvlistitem{Experienced with developing and deploying software on UNIX systems.}


\end{document}
