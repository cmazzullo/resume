% resume.tex
%
% This is the definitive source code of my resume, all of the other
% files in this folder are compiled from it.

\documentclass[11pt,a4paper,sans,colorlinks,urlcolor=blue]{moderncv}
\moderncvstyle{banking} % style options: 'casual', 'classic', 'oldstyle', 'banking'
\moderncvcolor{blue}
\usepackage[scale=0.75]{geometry} % adjust the page margins


% PERSONAL DATA

\name{Christopher}{Mazzullo}
\phone{(202)~253~1258}
\email{chris.mazzullo@gmail.com}
\homepage{github.com/cmazzullo}


% CONTENT

\begin{document}

\makecvtitle


\section{Objective}
\cvitem{} { Software engineer with several years of experience
  building and maintaining web sites looking to ...OBJECTIVE GOES HERE...  }

\section{Experience}

\cventry{May 2017--Present}{Web Developer}{IMS, Inc.}{Calverton, MD}{}{
  \begin{itemize}
  \item Developed and maintained websites serving healthcare datasets
    to researchers including
    % Wrap links in \mbox to prevent line breaks inside them
    \mbox{\href{biolincc.nhlbi.nih.gov}{biolincc.nhlbi.nih.gov}},
    \mbox{\href{repository.niddk.nih.gov}{repository.niddk.nih.gov}}
    and \mbox{\href{neurobiobank.nih.gov}{neurobiobank.nih.gov}} using
    a stack based on Django, Elasticsearch, and PostgreSQL.
  \item Led the conversion of
    \mbox{\href{biolincc.nhlbi.nih.gov}{biolincc.nhlbi.nih.gov}} and
    \mbox{\href{repository.niddk.nih.gov}{repository.niddk.nih.gov}}
    from Python 2 to Python 3.
  \item Started a series of internal technical talks to help spread
    organizational knowledge and new techniques among IMS's web
    developers.
  \item Guided new features all the way from requirements-gathering
    through implementation, testing and release.
  \end{itemize}
}

\cventry{Sept 2015--Nov 2016}{Programmer}{US Naval Observatory}{Washington, DC}{}{
  \begin{itemize}
  \item Maintained and improved Fortran 95 codebases responsible for computing
    Earth orientation parameters, the results of which are used in Navy GPS systems.
  \item Updated the public facing websites of the Earth Orientation
    division at USNO, and automated processes using Python and Bash to
    generate its data products.
\end{itemize}}

\cventry{June 2014--Sep 2015}{Python Developer}{US Geological Survey}{Reston, VA}{}{
  \begin{itemize}
  \item Designed algorithms using Fourier analysis to convert pressure
    data to wave height and frequency statistics, repository available here:
    \mbox{\href{http://github.com/cmazzullo/wave-sensor}{github.com/cmazzullo/wave-sensor}}.
  \item Developed programs to clean, interpret and visualize large
    NetCDF files of raw data from pressure sensors showing trends in
    wave height and frequency on the Atlantic coast.
\end{itemize}}

\cventry{May 2014--May 2015}{Undergraduate Researcher}{George Mason University}{Fairfax, VA}{}{
  \begin{itemize}
  \item Developed software in IDL to analyze and visualize solar data
    from the Hinode satellite with funding from the Undergraduate
    Research Scholars Program, an internal grant at Mason.
  \item Investigated the velocity of plasma in coronal loops to help
    discriminate between different models of coronal heating, and
    presented the research at the 2015 National Conference on
    Undergraduate Research.
\end{itemize}}

\section{Education}
\cventry{May 2015}{\emph{BS} Physics, Minor Computer Science}{George Mason University}{Fairfax, VA}{}{}

\end{document}
